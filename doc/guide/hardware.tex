\chapter{Hardware}


\section{Introduction}\label{introduction}

An hour spent checking your schematic will avoid many hours of testing
grief!

\subsection{Recommendations}\label{recommendations}

\begin{enumerate}
\item
  Independently fuse the buck converter and H-bridge. This allows a fuse
  to be removed to isolate part of the circuit when finding a fault,
  such as a short across the power rails.

\item
  Have a zener diode to protect against overvoltage input from a bench
  power supply.

\item
  Have current limiting resistors for all off-board signals.

\item
  Have plenty of testpoints, especially for power supplies and signals.
  You will never have too many!

\item
  Have at least one grunty ground testpoint for attaching a scope ground
  clip.

\item
  Have a dedicated PIO pin to drive a testpoint that you can use to
  trigger an oscilloscope for debugging.

\item
  Have the SAM4S erase pin connected to a test point close to a 3.3 V
  testpoint. This is useful to completely erase the SAM4S flash memory
  when nothing works.
\end{enumerate}

\section{SAM4S MCU}\label{sam4s-mcu}

The SAM4S MCU is overkill for this assignment but is typical of
ARM processors used for bare-metal applications.

\subsection{Power pins}\label{power-pins}

The SAM4S has four grounds. They must all be connected. There are also
seven power pins. These must all be connected since they power different
parts of the chip. Note, some pins require 3.3 V while others require
1.2 V. The 1.2 V is generated by an internal voltage regulator.

\subsection{Peripheral pins}\label{peripheral-pins}

\textbf{Many of the peripheral pins are dedicated and cannot be
reassigned in software}, e.g., SPI, TWI, and USB pins. Note, there are
restrictions on the PWM pins.

By default the \pin{PB4} and \pin{PB5} pins are configured for the JTAG debugger.
These can be used for general PIO after setting an internal bit.  See
\protect\hyperref[disabling-jtag-pins]{disabling JTAG pins}.

The logic levels are set by the voltage on the VDDIO pin (usually
3.3\,V).  \pin{PA12}--\pin{PA14} and \pin{PA26}--\pin{PA31} can sink/source
4\,mA.  The USB pins (\pin{PB10}--\pin{PB11}) can sink/source 30\,mA.  The
rest can only sink/source 2\,mA of current


\subsection{USART/UART}\label{usartuart}

The SAM4S has two USARTs and two UARTs. The USARTs can emulate a
UART, have hardware flow control, and have a better driver so they are
recommended if you need a UART interface.


\subsection{PWM}\label{pwm}

The SAM4S can generate four independent PWM signals. There are
restrictions on which SAM4S pins they come out on. Note, the PWMLx and
PWMHx signals are complementary.

\subsection{TWI}\label{twi}

The SAM4S has two TWI peripherals (that can act as a master and
a slave) with dedicated TWD and TWCK pins. External pull-up resistors
are required.  TWI1 shares pins with JTAG; you will need to disable
JTAG in software.

\subsection{SPI}\label{spi}

The SAM4S has a single SPI peripheral with dedicated SCK, MISO,
and MOSI pins. Any PIO pin can be used for the chip
select\footnote{For high speed operation, you should use on of the
  dedicated chip select pins.}.

\subsection{ADC}\label{adc}

The SAM4S has a single ADC with a multiplexer to select one of a
number of analogue inputs.  It can sample at 1\,MHz.

\subsection{USB}\label{usb}

The SAM4S has a single USB peripheral connected to the DDP and DDM
pins. 27 ohm series termination resistors are required, placed close to
the SAM4S.

\section{Other chips}\label{other-chips}

\subsection{DRV8833 dual H-bridge}\label{drv8833-dual-h-bridge}

The capacitor connected to the bootstrap pin must be rated for 16 V. The
datasheet recommends an X7R dielectric.

If you want control over fast decay and slow decay in both forward and
reverse you will need four PWM signals. The SAM4S can provide four
independent PWM signals but be careful that PWMxH and PWMxL are driven
with the same PWM signal but are complementary.

You can drive the H-bridge with only two PWM signals but you will have
fast decay in one direction and slow decay in the other. It appears that
fast decay is better at slow speeds since the motor heats less.

\subsection{NRF24L01+ radio}\label{nrf24l01-radio}

This operates around 2.4 GHz and has 128 programmable channels, each of
1 MHz. A 5 byte address is appended to the start of each transmission
and the receiver will only respond when the address matches.

The radio interfaces to the SAM4S using the SPI bus. The IRQ
pin is driven low to indicate a packet has been received.

A well filtered power supply is critical otherwise the range will be
limited on reception. Preferably, the device should have its own 3V3
regulator with a low-pass RC filter comprised of a series resistor and
large capacitor (say 22 uF) or a low-pass LC filter made with a ferrite
bead and capacitor.

\subsection{MPU9250 IMU}\label{mpu9250-imu}

This contains a three axis accelerometer, a three axis gyroscope, and a
three axis magnetometer. It appears that the magnetometer has been
bolted on to the accelerometers/gyroscopes and requires more hoop
jumping in software to make it work.

It has two different I2C addresses (0X68 and OX69) depending on
the state of the AD0 pin.


\subsection{Voltage regulators}\label{voltage-regulators}

There are many flavours of \href{voltage_regulators}{voltage regulator}.
Some are better for digital applications, some are better for analogue
applications, some are better for low power applications, etc.

If you are using a voltage regulator with an enable pin, do not forget
to allow for the time for the output voltage to ramp up. This can be
tens of milliseconds depending on the capacitive load and current draw.

Note, some regulators have pins that you must not connect. Some have
multiple pins for the same purpose; these must all be connected.

\section{PCB recommendations}\label{pcb-recommendations}

\subsection{Placement}\label{placement}

\begin{enumerate}
\item
  Keep small signal analogue components (radio) well away from digital
  electronics and power electronics.
\item
  Place local decoupling capacitors to minimise the loop area.
\item
  Keep the crystal close to the MCU.
\item
  Place switches so they can be pushed.
\item
  Place LEDs so they can be seen.
\item
  Place USB connector so it can be connected to.
\item
  Place connectors on edge with wires going away from the board.
\end{enumerate}

\subsection{Check list}\label{check-list}

\begin{enumerate}
\item
  No planes under the radio antenna.
\item
  The SPI signals for the radio are connected to the correct MCU pins.
\item
  The PWM signals for the motor are connected to the correct MCU pins.
  Note, the PWMLx and PWMHx pins are complementary and cannot be driven
  independently.
\item
  All the MCU VDD pins need to be powered.
\item
  All the MCU GND pins need to be powered.
\item
  Avoid connecting to \pin{PB4} and \pin{PB5} (say for TWI1).  If you do you will need to disable JTAG.
\end{enumerate}

