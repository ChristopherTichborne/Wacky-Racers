\chapter{OpenOCD}

The Open On-Chip Debugger (OpenOCD) is an open-source on-chip debugging,
in-system programming, and boundary-scan testing tool. It is able to
communicate with various ARM and MIPS microprocessors via \url{JTAG} or
\url{SWD}. It works with a number of different \url{JTAG} or \url{SWD}
interfaces/programmers. User interaction can be achieved via telnet or
the GNU debugger (GDB).

\section{Configuration files}
\label{configuration-files}

OpenOCD needs a \wikiref{OpenOCD_configuration}{configuration file} to
specify the interface (USB or parallel port) and the target system.
Unfortunately, these change with every new release of OpenOCD.

\section{Running OpenOCD}
\label{running-openocd}

OpenOCD runs as a daemon program in the background and can be controlled
from other programs using TCP/IP sockets. This means that you can
remotely debug from another computer. The socket ports it uses are
specified in the \wikiref{OpenOCD_configuration}{OpenOCD configuration}
file supplied when it starts. By default, OpenOCD uses port 3333 for
\url{gdb} and port 4444 for general interaction using Telnet.

\subsection{Communicating with OpenOCD using telnet}
\label{communicating-with-openocd-using-telnet}

If OpenOCD is running, commands can be sent to it using telnet. For
example,

\begin{minted}{bash}
$ telnet localhost:4444
> monitor flash info 0
\end{minted}

\subsection{Communicating with OpenOCD using GDB}
\label{communicating-with-openocd-using-gdb}

If OpenOCD is running, commands can be set to it with \url{gdb}. There
are two steps: connecting to OpenOCD with the target remote command, and
then sending a command with the GDB monitor command. For example,

\begin{minted}{bash}
$ gdb
(gdb) target remote localhost:3333
(gdb) monitor flash info 0
\end{minted}

\section{OpenOCD commands}
\label{openocd-commands}

All the gory OpenOCD details can be found in the
\wikiref{Media:openocd.pdf}{OpenOCD manual}. If you are getting strange
errors see \wikiref{OpenOCD_errors}{OpenOCD errors}.

\section{Flash programming}
\label{flash-programming}

OpenOCD can program the flash program memory from a binary file.

\section{FTDI support}
\label{ftdi-support}

Many \wikiref{JTAG_interface}{JTAG interfaces} use chips made by Future
Technology Devices International (FTDI) to translate between the USB and
JTAG protocols. FTDI provide a library to communicate with these
devices; however, this is not open-source and hence cannot be
distributed. There is an open-source equivalent,
\href{http://freshmeat.net/projects/libftdi/}{libftdi}, which in turn
uses the open-source libusb. This can be temperamental to get working on
a Windows system.

\section{Getting OpenOCD}
\label{getting-openocd}

The developers of OpenOCD release source packages (and provide
\url{subversion} access) but do not provide official builds for any
operating system. However, there are various sites which provide
prebuilt versions of OpenOCD.

\subsection{Windows}
\label{windows}

Windows installers for release versions of OpenOCD can be found
\href{http://www.freddiechopin.info/index.php/en/download/category/4-openocd}{here}.
Some development versions can be found
\href{http://www.freddiechopin.info/index.php/en/download/category/10-openocd-dev}{here}.
Unless you need a feature not present in the release version, avoid
getting the development versions as they can be less stable.

Alternatively, the
\href{http://www.siwawi.arubi.uni-kl.de/avr_projects/arm_projects/}{WinARM}
toolchain contains a version of OpenOCD, albeit one that appears (from
the description on the project page) to be out of date. It is also
possible to build your own copy of OpenOCD using either
\href{http://www.mingw.org/}{MinGW} or
\href{http://www.cygwin.com/}{Cygwin}; instructions for this are posted
in various places online.

Realistically, it's probably easier to set up a virtual machine with
Linux and use that instead.

\subsection{Linux}
\label{linux}

Many recent distributions of Linux have OpenOCD in their software
repositories. However, this is often out of date and (in some cases)
buggy. In this case it is straightforward to
\wikiref{Building_OpenOCD_under_Linux}{build the latest version}.

\subsection{Mac OSX}
\label{mac-osx}

The easiest way to install OpenOCD on Mac OSX is to use
\href{http://brew.sh/}{Homebrew}. If you wish to use JTAG with an
adaptor based on a FTDI chip (for example, the UCECE USB to JTAG
adaptor or a Bus Blaster), make sure you include libftdi support by
installing with the command:
%
\begin{minted}{bash}
$ brew install openocd --enable-ft2232_libftdi
\end{minted}

\section{Errors}
\label{errors}

See \url{http://ecewiki.elec.canterbury.ac.nz/mediawiki/index.php/OpenOCD_errors}.
