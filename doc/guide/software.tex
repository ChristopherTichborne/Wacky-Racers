\chapter{Software}

\section{Preparation}
\label{preparation}

Do not underestimate the effort required to flash your first LED. You
require:

\begin{itemize}
\item
  A computer with \program{git} installed and a useful shell program such as
  \program{bash}. UC has a \href{http://ucmirror.canterbury.ac.nz/}{mirror}
  for a variety of Linux distributions; we recommend Ubuntu or Mint.
\item
  A working ARM toolchain (arm-none-eabi-gcc/g++ version 4.9.3 or newer).
\item
  \program{OpenOCD} V0.8.0 or later.
\item
  An ST-link programmer and 10-wire ribbon cable for programming. You
  can get the adaptor from Scott Lloyd in the SMT lab. You will need
  to make your own cable (For a grey ribbon cable, align the red
  stripe with the small arrow denoting pin 1 on the connector. For a
  rainbow ribbon cable, connect the brown wire to pin 1.). There are
  two variants of the ST-link programmer with \textbf{different
    pinouts} so you may need to customise your programming cable.
\item
  Plenty of gumption.
\end{itemize}

\subsection{Toolchain}
\label{toolchain}

The required software is installed on computers in the ESL and
CAE. It should run under both Linux and Windows. If there is a
problem ask the technical staff.

If you are working from a personal Linux computer make sure to update
your machine to the latest software versions. This is particularly
important for the ARM toolchain (arm-none-eabi-gcc/g++) which should be
at version 4.9.3 or newer. You can do this with the following command
for an Ubuntu or Mint distribution:

\begin{minted}{bash}
$ sudo apt update && sudo apt upgrade}
\end{minted}

For MacOS machines that have \href{https://brew.sh}{homebrew} installed,
you can use the following command:

\begin{minted}{bash}
$ brew install openocd git}
$ brew cask install gcc-arm-embedded}
\end{minted}

\subsection{Project forking}
\label{project-forking}

The example software is hosted on the eng-git \program{Git} server at
\href{https://eng-git.canterbury.ac.nz/wacky-racers/wacky-racers-2021}{\url{https://eng-git.canterbury.ac.nz/wacky-racers/wacky-racers-2021}}.
Your group leader should have forked this template project. This creates
your own group copy of the project on the eng-git server that you can
modify, add members, etc. To fork the project:

\begin{enumerate}
\item
  Go to
  \href{https://eng-git.canterbury.ac.nz/wacky-racers/wacky-racers-2021}{\url{https://eng-git.canterbury.ac.nz/wacky-racers/wacky-racers-2021}}.
\item
  Click the `Fork' button. This will create a copy of the main repository
  for the project.
\item
  Click on the `Settings' menu then click the `Expand' button for
  `Sharing and permissions'. Change `Project Visibility' to `Private'.
\item
  Click on the `Members' menu and add group members as Developers.
\end{enumerate}

\subsection{Project cloning}
\label{project-cloning}

Once your project has been forked from the template project, each group
member needs to clone it. This makes a local copy of your project on
your computer. 

If you are using an ECE computer, it is advised that you clone the
project on to a removable USB flash drive. This will make git operations
and compilation a 100 times faster than using the networked file system.

There are two ways to clone the project. If you are impatient and do not
mind having to enter a username and password for every git pull and push
operation use:
%
\begin{minted}[breaklines]{bash}
$ git clone https://eng-git.canterbury.ac.nz/groupleader-userid/wacky-racers-2021.git wacky-racers
\end{minted}

Otherwise, set up \program{ssh-keys} and use:
%
\begin{minted}[breaklines]{bash}
$ git clone git@eng-git.canterbury.ac.nz:groupleader-userid/wacky-racers-2021.git wacky-racers
\end{minted}

You can several different cloned copies of your project in different
directories. Sometimes if you feel that the world, and \program{git} in
particular, is against you, clone a new copy, using:


\begin{minted}[breaklines]{bash}
$ git clone https://eng-git.canterbury.ac.nz/groupleader-userid/wacky-racers-2021.git wacky-racers-new
\end{minted}

\subsection{Configuration}
\label{configuration}

Each board has different PIO definitions and requires its own
configuration information. The \wfile{boards}
directory contains a number of configurations; one for the hat and racer
boards. Each configuration directory contains three files:

\begin{itemize}
\item
  \file{board.mk} is a makefile fragment that specifies the MCU model,
  optimisation level, etc.
\item
  \file{target.h} is a C header file that defines the PIO pins and
  clock speeds.
\item
  \file{config.h} is a C header file that wraps target.h. It's purpose
  is for porting to different compilers.
\end{itemize}

\textbf{You will need to edit the target.h file for your board} and set
the definitions appropriate for your hardware. Here's an excerpt from
\file{target.h} for a hat board:

\begin{minted}{C}
/* USB  */
#define USB_VBUS_PIO PA5_PIO

/* ADC  */
#define ADC_BATTERY PB3_PIO
#define ADC_JOYSTICK_X PB2_PIO
#define ADC_JOYSTICK_Y PB1_PIO

/* IMU  */
#define IMU_INT_PIO PA0_PIO

/* LEDs  */
#define LED1_PIO PA20_PIO
#define LED2_PIO PA23_PIO
\end{minted}

\section{First program}
\label{first-program}

Your first program to test your board should only flash an LED (the
hello world equivalent for embedded systems). The key to testing new
hardware is to have many programs that only do one simple task each.

\subsection{OpenOCD}
\label{openocd}

\program{OpenOCD} is used to program the SAM4S. For this assignment,
we are using a ST-link programmer to connect to the SAM4S using serial
wire debug (SWD). This connects to your board with a 10-wire ribbon
cable and an IDC connector.

\begin{enumerate}
\item
  Before you start, disconnect the battery and other cables from your
  PCB.
\item
  Connect a 10-wire ribbon cable from the ST-link programmer to the
  programming header on your PCB. This will provide 3.3 V to your
  board so your green power LED should light.
\item
  Open a \textbf{new terminal window} and start \program{OpenOCD}.
\end{enumerate}

\begin{minted}{bash}
$ cd wacky-racers
$ openocd -f src/mat91lib/sam4s/scripts/sam4s_stlink.cfg
\end{minted}

All going well, the last line output from \program{OpenOCD} should be:

\begin{verbatim}
Info : sam4.cpu: hardware has 6 breakpoints, 4 watchpoints
\end{verbatim}

Congrats if you get this! It means you have correctly soldered your
SAM4S. If not, do not despair and do not remove your SAM4S. Instead,
see \protect\hyperref[troubleshooting]{troubleshooting}.


\subsection{LED flash program}
\label{led-flash-program}

For your first program, use
\wfile{test-apps/ledflash1/ledflash1.c}. The macros
\code{LED1\_PIO} and \code{LED2\_PIO} need to be defined in
\file{target.h} (see
\protect\hyperref[configuration]{configuration}). If your LEDs are
active high, replace 1 with 0 in the LED config structures.

\inputminted{C}{../../src/test-apps/ledflash1/ledflash1.c}

\subsection{Compilation}
\label{compilation}

Due to the many files required, compilation is performed using
makefiles.

The demo test programs are generic and you need to specify which board
you are compiling them for. The board configuration file can be chosen
dynamically by defining the environment variable \code{BOARD}. For
example:
%
\begin{minted}{bash}
$ cd src/test-apps/ledflash1
$ BOARD=racer make
\end{minted}

If all goes well, you should see at the end:
%
\begin{verbatim}
   text    data     bss     dec     hex filename
  19432    2420    2732   24584    6008 ledflash1.bin
\end{verbatim}

To avoid having to specify the environment variable \code{BOARD}, you
can define it for the rest of your session using:
%
\begin{minted}{bash}
$ export BOARD=racer
\end{minted}
%
and then just use:
%
\begin{minted}{bash}
$ make
\end{minted}

Note, if you compile with the \code{BOARD} environment variable set incorrectly, use:
%
\begin{minted}{bash}
$ make clean
$ make  
\end{minted}
%
to delete all the object files and rebuild.


\subsection{Booting from flash memory}
\label{booting-from-flash-memory}

By default the SAM4S runs a bootloader program stored in ROM. The SAM4S
needs to be configured to run your application in flash memory.

If \program{OpenOCD} is running you can do this with:

\begin{minted}{bash}
$ make bootflash
\end{minted}

Unless you force a complete erasure of the SAM4S flash memory by
connecting the \pin{ERASE} pin to 3.3 V, you will not need to repeat
this command.

\subsection{Programming}
\label{programming}

If \program{OpenOCD} is running you can store your program in the flash
memory of the SAM4S using:

\begin{minted}{bash}
$ make program
\end{minted}

When this finishes, one of your LEDs should flash. If so, congrats! If
not, see \protect\hyperref[troubleshooting]{troubleshooting}.

To reset your SAM4S, you can use:
%
\begin{minted}{bash}
$ make reset
\end{minted}

\section{USB interfacing}
\label{usb-interfacing}

To help debug your programs, it is wise to set up USB CDC. For
example, here's the code for
\wfile{test-apps/usb_serial_test1/usb_serial_test1.c}.

\inputminted{C}{../../src/test-apps/usb_serial_test1/usb_serial_test1.c}

Note, the string \code{"/dev/usb\_tty"} is used to name the USB serial
device on the SAM4S so that it can be used by the C stdio function
\code{freopen}.   

To get this programme to work you need to compile it and program the SAM4S using:

\begin{minted}{bash}
$ cd wacky-racers/src/test-apps/usb_serial_test1
$ make program
\end{minted}

You then need to connect your computer to the USB connector on your PCB.
If you are running Linux, run:
%
\begin{minted}{bash}
$ dmesg
\end{minted}

This should say something like:
%
\begin{verbatim}
Apr 30 11:03:50 thing4 kernel: [52704.481352] usb 2-3.3: New USB device found, idVendor=03eb, idProduct=6202
Apr 30 11:03:50 thing4 kernel: [52704.481357] usb 2-3.3: New USB device strings: Mfr=1, Product=2, SerialNumber=3
Apr 30 11:03:50 thing4 kernel: [52704.482060] cdc_acm 2-3.3:1.0: ttyACM0: USB ACM device
\end{verbatim}

Ignore the errors for now\footnote{They have popped up with new Linux
  kernels and we have yet to determine why.}. If you see
\texttt{ttyACM0:\ USB\ ACM\ device} congrats. If not, see \wikiref{USB_debugging}{USB debugging}.

You can now run a \wikiref{Serial_terminal_applications}{serial
  terminal program}. For example, on Linux:

\begin{minted}{bash}
$ gtkterm -p /dev/ttyACM0
\end{minted}

All going well, this will repeatedly print 'Hello world'.

If you get an error 'device is busy', it is likely that the ModemManager
program has automatically connected to your device on the sly. This
program should be disabled on the ECE computers. For more about this and
using other operating systems, search for USB CDC on ecewiki.

\section{Test programs}
\label{test-programs}

There are a number of test programs in the directory
\wfile{test-apps}. Where possible these are written to be
independent of the target board using configuration files (see
\protect\hyperref[configuration]{configuration}).

\subsection{PWM test}
\label{pwm-test}

The program \wfile{test-apps/pwm_test2/pwm_test2.c} provides an
example of driving PWM signals. Notes:
%
\begin{enumerate}
\item
  This is for a different H-bridge module that requires two PWM signals
  and forward/reverse signals. You will need to generate four PWM
  signals or be clever with two PWM signals.
\item
  The \code{pwm\_cfg\_t} structure configures the frequency, duty
  cycle and alignment of the output PWM.
\end{enumerate}

The most likely problem is that you have not used a PIO pin that can
be driven as a PWM output. The SAM4S can generate four independent
hardware PWM signals. See \wfile{mat91lib/pwm/pwm.c} for a list of
supported PIO pins. Note, \pin{PA16}, \pin{PA30}, and \pin{PB13} are
different options for PWM2.

To drive the motors you will need to use a bench power supply. Start
with the current limit set at 100\,mA maximum in case there are any
board shorts.  When all is well, you can increase the current limit.

\subsection{IMU test}
\label{imu-test}

The program \wfile{test-apps/imu_test1/imu_test1.c} provides an
example of using the MPU9250 IMU.  All going well, this prints three
16-bit acceleration values per line to USB CDC. Tip your board over,
and the the third (z-axis) value should go negative since this
measures the effect of gravity on a little mass inside the IMU pulling
on a spring.

If you get 'Cannot detect IMU' the main reasons are:

\begin{enumerate}
\item
  You have specified the incorrect address. Use \code{MPU\_0} if the
  AD0 pin is connected to ground otherwise use \code{MPU\_1}.
\item
  You are using TWI1. The \pin{PB4} and \pin{PB5} pins used by TWI1 default to JTAG
  pins. See \protect\hyperref[disabling-jtag-pins]{disabling JTAG pins}.
\item
  You do not have TWI/I2C pull-up resistors connected to 3V3 or the
  wrong resistor values. Use a scope to look at the signals \pin{TWCK/SCL}
  and \pin{TWD/SDA}.
\item
  Check that the auxiliary I2C bus signals are not connected
  (otherwise the magnetometer will not respond).  
\item
  You have a soldering problem. Try pressing on a side of the IMU with a
  fingernail to see if it starts working.
\end{enumerate}

If you reset the SAM4S in the middle of a transaction with the
IMU, the IMU gets confused and holds the \pin{TWD/SDA} line low. This
requires recycling of the power or sending out some dummy clocks on
the \pin{TWCK/SCL} signal.


\subsection{Radio test}
\label{radio-test}

The program \wfile{test-apps/radio_tx_test1/radio_tx_test1.c} provides
an example of using the radio as a transmitter.

The companion program
\wfile{test-apps/radio_rx_test1/radio_rx_test1.c} provides an example
of using the radio as a receiver.

Notes:
%
\begin{enumerate}
\item
  Both programs must use the same RF channel and the same address. Some
  RF channels are better than others since some overlap with WiFi
  and Bluetooth. The address is used to distinguish devices
  operating on the same channel. Note, the transmitter expects an
  acknowledge from a receiver on the same address and channel.
\item
  The radio `write` method blocks waiting for an auto-acknowledgement
  from the receiver device. This acknowledgement is performed in
  hardware. If no acknowledgement is received, it retries for up to 15
  times. The auto-acknowledgement and number of retries can be
  configured in software.
\end{enumerate}

If you cannot communicate between your hat and racer boards, try
communicating with the radio test modules Scott Lloyd has in the SMT
lab.

\section{Your hat/racer program}
\label{your-hatracer-program}

There is skeleton code, see
\wfile{wacky-apps/hat/hat.c} and
\wfile{wacky-apps/racer/racer.c}. We recommend that
build your programs incrementally and that you poll your devices with a
paced loop and not use interrupts. It is a good idea to disable the
watchdog timer until you have robust code.

\begin{minted}{C}
#include <stdio.h>
#include <math.h>
#include "mcu.h"
#include "pacer.h"

// Set to the desired polling rate.
#define PACER_RATE 1000

int
main (void)
{
    // If you are using PB4 or PB5 (say for TWI1) uncomment the next line.
    // mcu_jtag_disable ();    

    pacer_init (PACER_RATE);    
    
    mcu_watchdog_enable ();

    while (1)
    {
        pacer_wait ();

        mcu_watchdog_reset ();

        /* Do your stuff here...  */
        
    }
    
    return 0;
}
\end{minted}

\section{Debugging}
\label{debugging}

If \program{OpenOCD} is running, a running program can be debugged using:

\begin{minted}{bash}
$ make debug
\end{minted}

This starts \program{GDB} and attaches to the target MCU. \program{GDB} is a
command line debugger but there are many GUI programs that will control
it, for example, \program{vscode} and \program{geany} have plugins.

\program{GDB} allows you to inspect the CPU registers, memory, set
breakpoints, set watchpoints, and much more. The only gnarly thing is
that when compiling with optimisation, the compiler may reorder
statements or even remove them. Debugging embedded systems is also made
difficult by the asynchronous nature of I/O.

You can reset your program using the GDB `jump reset` command. However,
this does not reset the peripherals as with a power-on reset.

\section{Howtos}
\label{howtos}

\subsection{Disabling JTAG pins}
\label{disabling-jtag-pins}

By default \pin{PB4} and \pin{PB5} are configured as JTAG pins. You can turn
them into PIO pins or use them for TWI1 using:
%
\begin{minted}{C}
#include "mcu.h"

void main (void)
{
    mcu_jtag_disable ();
}
\end{minted}

\subsection{Watchdog timer}
\label{watchdog-timer}

The watchdog timer is useful for resetting the SAM4S if it hangs in a
loop.  It is disabled by default but can be enabled using:
%
\begin{minted}{C}
#include "mcu.h"

void main (void)
{
    mcu_watchdog_enable ();
   
    while (1)
    {
         /* Do your stuff here.  */

         mcu_watchdog_reset ();
    }
}
\end{minted}

\section{Under the bonnet}
\label{under-the-bonnet}

\file{mmculib} is a library of C drivers, mostly for performing high-level I/O.
It is written to be microcontroller neutral.

\file{mat91lib} is a library of C drivers specifically for interfacing
with the peripherals of Atmel AT91 microcontrollers such as the Atmel
SAM4S. It provides the hardware abstraction layer.

The building is controlled by \wfile{mat91lib/mat91lib.mk}. This is a
makefile fragment loaded by \wfile{mmculib/mmculib.mk}.
\file{wacky-reacers/src/mat91lib/mat91lib.mk} loads other makefile fragments for each
peripheral or driver required. It also automatically generates
dependency files for the gazillions of other files that are required to
make things work.

\section{Git}
\label{git}

To properly use git you should commit and push often. The smaller the
changes and the more often you make per commit, the smaller the chance
of the dreaded merge conflict.

\subsection{Git pulling from upstream}
\label{git-pulling-from-upstream}

To be able to get updates if the template project is modified you will
need to:

\begin{minted}{bash}
$ cd wacky-racers 
$ git remote add upstream https://eng-git.canterbury.ac.nz/wacky-racers/wacky-racers-2021.git  
\end{minted}

Again if you do not want to manually enter your password (and have
ssh-keys uploaded) you can use:
%
\begin{minted}[breaklines]{bash}
$ cd wacky-racers 
$ git remote add upstream git@eng-git.canterbury.ac.nz:wacky-racers/wacky-racers-2021.git
\end{minted}

Once you have defined the upstream source, to get the updates from the
main repository use:
%
\begin{minted}{bash}
$ git pull upstream master
\end{minted}

If you enter the wrong URL make a mistake, you can list the remote
servers and delete the dodgy entry using:

\begin{minted}{bash}
$ git remote -v
$ git remote rm upstream
\end{minted}

Note, \textbf{origin} refers to your group project and \textbf{upstream}
refers to the template project that origin was forked from.

\subsection{Git merging}
\label{git-merging}

The bane of all version control programs is dealing with a merge
conflict. You can reduce the chance of this happening by committing and
pushing faster than other people in your group.

If you get a message such as:

\begin{verbatim}
From https://eng-git.canterbury.ac.nz/wacky-racers/wacky-racers-2021
 * branch            master     -> FETCH_HEAD
error: Your local changes to the following files would be overwritten by merge:
    src/test-apps/imu_test1/imu_test1.c
Please, commit your changes or stash them before you can merge.
\end{verbatim}

what you should so is:

\begin{minted}{bash}
$ git stash
$ git pull
$ git stash pop
# You may now have a merge error.  You will now have to edit the offending file, in this case imu_test1.c
# Once the file has been fixed
$ git add src/test-apps/imu_test1/imu_test1.c
$ git commit -m "Fix merge"
\end{minted}

Sometimes when you do a git pull you will be thrown into a text editor
to type a merge comment. The choice of editor is controlled by an
environment variable \code{EDITOR}. On the ECE computers this defaults
to emacs. You can changed this by adding a line such as the following to
the \file{.bash\_profile} file in your home directory.

\begin{minted}{bash}
$ export EDITOR=geany
\end{minted}

By the way, to exit emacs type ctrl-x ctrl-c, to exit vi or vim type \code{:q!}

\textbf{Please do not edit the files in the mat91lib, mmculib, and
wackylib} directories since this can lead to merge problems in the
future. If you find a bug or would like additional functionality let MPH
or one of the TAs know.
