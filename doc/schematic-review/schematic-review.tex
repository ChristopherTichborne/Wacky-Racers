\documentclass[a4paper, 12pt]{article}
\input a4size

\title{ENCE461 Schematic Review}
\author{M.P. Hayes}
\date{}

\begin{document}
\maketitle


\begin{center}
\textbf{Bring your schematics, printed on A3 paper}  
\end{center}


\section{Common}

\begin{enumerate}
\item Student names and group number in title block

\item Battery fusing (this is mandatory)

\item Can be powered from USB

\item Use serial wire debug interface for programming  

\item 3.3\,V MCU regulator can be back driven

\item Short circuit protection for MCU pio pins going to external headers

\item Battery voltage monitoring

\item Do the analogue inputs to the MCU exceed 3.3 V?  

\item LEDs for debugging  

\item Jumpers for mode configuration

\item Pullup resistors on TWI bus  
  
\item Test points

\item Ground test points

\item Game board interface connects to USART (TXD0/PA6 or TXD1/PA22 to
  TXD, RXD0/PA5 or RXD1/PA21 to RXD)

\item USB has series termination resistors

\item VBUS detection through voltage divider to PIO pin.  This is
  needed so that the MCU can tell when USB is plugged in or removed.
  You will also need diodes (or jumpers) so that the USB 5\,V can be
  connected to the 5\,V from the switching regulator.

\item Power supply filtering for radio (recommend ferrite bead or
  resistor in series with power rail with parallel capacitor)

\item The radio needs to be connected to SPI pins (MISO/PA12,
  MOSI/PA13, SCK/PA14)

\item TWI uses TWCK0/PA4 and TWD0/PA3 or TWCK1/PB5 and TWD1/PB4.
  
\item SAM4S erase pin on testpoint

\item SAM4S has 12\,MHz crystal

\item Reset button connected to NRST pin

\item Power on/off button connected to WKUPn pin

\item Avoid PB4--PB5 for general I/O (they default to JTAG pins on
  reset but can be reconfigured in software)

\item Have external pull-down resistors to ensure chips are disabled on
  power-up

\item Have a few spare PIO pins connected to pads for last minute mods.  
  
\end{enumerate}


\section{Hat board}

\begin{enumerate}
\item Battery can be charged from USB
  
\item Fall-back option if IMU does not work

\item Nav-switch or joystick for remote control  
  
\item Drive circuit for piezo tweeter


\end{enumerate}

\section{Racer board}


\begin{enumerate}
\item Fall-back option to drive motors via servo interface using PWM
  if H-bridge driver fails
  
\item MOSFET(s) for actuator (if use p-channel MOSFET need transistor
  to provide sufficient gate voltage to turn MOSFET off)

\item H-bridge driven by four PWM signals (it is best to use PWMHx,
  note PWMLx and PWMHx are complementary)

\item H-bridge AISEN and BISEN pins connected to ground (unless using
  current control)

\end{enumerate}


\end{document}
